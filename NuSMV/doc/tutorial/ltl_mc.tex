% -*-latex-*-
\section{Linear Temporal Logic}
\label{Linear Temporal Logic}
\index{Linear Temporal Logic}

\nusmv allows for specifications expressed in LTL. Intuitively,
while CTL specifications express properties over the computation tree of
the FSM (branching-time approach), LTL characterizes each linear path
induced by the FSM (linear-time approach). The two logics have in
general different expressive power, but also share a significant
intersection that includes most of the common properties used in
practice.
\index{LTL Operators}
Typical LTL operators are:
\begin{itemize}
\item \code{F p} (read ``in the future \code{p}''), 
stating that a certain condition \code{p} holds in one of the future
time instants;
\item \code{G p} (read ``globally \code{p}''), 
stating that a certain condition \code{p} holds in all future time 
instants;
\item \code{p U q} (read ``\code{p} until \code{q}''),
stating that condition \code{p} holds until a state is reached where
condition \code{q} holds;
\item \code{X p} (read ``next \code{p}''),
stating that condition \code{p} is true in the next state.
\end{itemize}
We remark that, differently from CTL, LTL temporal operators do not have
path quantifiers. In fact, LTL formulas are evaluated on linear paths,
and a formula is considered true in a given state if it is true for all
the paths starting in that state.


\section{Semaphore Example}
\label{LTL Semaphore Example}
\index{Examples!Semaphore!LTL}

\index{Specification!LTL}
Consider the case of the semaphore program and of the safety and
liveness properties already described in \cref{CTL Model Checking}. 
These properties correspond to LTL formulas
\begin{alltt}
G ! (proc1.state = critical & proc2.state = critical)
\end{alltt}
expressing that the two processes cannot be in the critical region
at the same time, and 
\begin{alltt}
G (proc1.state = entering -> F proc1.state = critical)
\end{alltt}
expressing that whenever a process wants to enter its critical session,
it eventually does.

\index{LTLSPEC keyword}
If we add the two corresponding LTL specification to the program, as
follows:\footnote{In \nusmv a LTL specification are introduced by
the keyword ``\code{LTLSPEC}''.}
\begin{alltt}
MODULE main
 VAR
   semaphore : boolean;
   proc1     : process user(semaphore);
   proc2     : process user(semaphore);
 ASSIGN
   init(semaphore) := FALSE;
 LTLSPEC G ! (proc1.state = critical & proc2.state = critical)
 LTLSPEC G (proc1.state = entering -> F proc1.state = critical)
MODULE user(semaphore)
 VAR
   state : \{idle, entering, critical, exiting\};
 ASSIGN
   init(state) := idle;
   next(state) :=
     case
       state = idle                  : \{idle, entering\};
       state = entering & !semaphore : critical;
       state = critical              : \{critical, exiting\};
       state = exiting               : idle;
       TRUE                          : state;
     esac;
   next(semaphore) :=
     case
       state = entering : TRUE;
       state = exiting  : FALSE;
       TRUE             : semaphore;
     esac;
 FAIRNESS
   running
\end{alltt}
\nusmv produces the following output:
\begin{alltt}
-- specification  G (!(proc1.state = critical & proc2.state = critical)) 
-- is true
-- specification  G (proc1.state = entering ->  F proc1.state = critical) 
-- is false
-- as demonstrated by the following execution sequence
Trace Description: CTL Counterexample
Trace Type: Counterexample
-> State: 1.1 <-
    semaphore = FALSE
    proc1.state = idle
    proc2.state = idle
-> Input: 1.2 <-
    _process_selector_ = proc2
-- Loop starts here
-> State 1.2 <-
[...]
\end{alltt}
That is, the first specification is true, while the second is false and
a counter-example path is generated.


\section{Past Temporal Operators}
\label{LTL Past Temporal Operators}
\index{Past Temporal Operators (LTL)}

In \nusmv, LTL properties can also include \emph{past} temporal
operators. Differently from standard temporal operators, that allow to
express properties over the future evolution of the FSM, past temporal
operators allow to characterize properties of the path that lead to the
current situation. The typical past operators are:
\begin{itemize}
\item \code{O p} (read "once \code{p}"), 
stating that a certain condition \code{p} holds in one of the past
time instants;
\item \code{H p} (read "historically \code{p}"), 
stating that a certain condition \code{p} holds in all previous time instants;
\item \code{p S q} (read "\code{p} since \code{q}"),
stating that condition \code{p} holds since a previous state where
condition \code{q} holds;
\item \code{Y p} (read "yesterday \code{p}"),
stating that condition \code{p} holds in the previous time instant.
\end{itemize}
\noindent
Past temporal operators can be combined with future temporal operators,
and allow for the compact characterization of complex properties.

A detailed description of the syntax of LTL formulas can be 
found in the \NuSMV User Manual.
%@node  Bounded Model Checking,  , LTL model checking, Tutorial
%  node-name,  next,  previous,  up
