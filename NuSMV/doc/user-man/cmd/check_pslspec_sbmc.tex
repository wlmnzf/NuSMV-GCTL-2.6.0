% -*-latex-*-
\begin{nusmvCommand}{check\_pslspec\_sbmc} {Performs SAT-based PSL model checking}

\cmdLine{check\_pslspec\_sbmc [-h] [-m | -o output-file] [-n number | -p
    \linebreak "\pslexpr [IN context]" | -P "name"] [-g] [-1] [-k\linebreak
    bmc\_lenght] [-l loopback]}

Check psl properties using SAT-based model checking. Use the SBMC algorithms.

A \pslexpr to be checked can be specified at command line using option
\commandopt{p}. Alternatively, option \commandopt{n} can be used for checking a
particular formula in the property database. If neither \commandopt{n} nor
\commandopt{p} are used, all the PSLSPEC formulas in the database are
checked. Options \commandopt{k} and \commandopt{l} can be used to define the
maximum problem bound, and the value of the loopback for the single generated
problems respectively; their values can be stored in the environment variables
\envvar{\it bmc\_lenght} and \envvar{\it bmc\_loopback}. Single problems can be
generated by using option \commandopt{1}. Bounded model checking problems can
be generated and dumped in a file by using option \commandopt{g}.

See variable \varName{use\_coi\_size\_sorting} for changing properties
verification order.

\begin{cmdOpt}
\opt{-m}{ Pipes the output generated by the command in processing
      \code{PSLSPEC}{s} to the program specified by the \envvar{PAGER}
      shell variable if defined, else through the \unix command
      \shellcommand{more}.}

\opt{-o \parameter{\filename{\it output-file}}}{Writes the output
      generated by the command in processing \code{PSLSPEC}s to the file
      \filename{\it output-file}}

\opt{-p \parameter{"\pslexpr\newline\hspace*{6mm} [IN context]"}}{ A PSL formula to be
      checked.  \code{context} is the module instance name which the
      variables in \pslexpr must be evaluated in.}

\opt{-n \parameter{\natnum{number}}}{ Checks the PSL property with
      index \natnum{number} in the property database.}

\opt{-P \parameter{\natnum{name}}}{Checks the PSL property named \natnum{name} in
the property database.} 

\opt{-g}{Dumps DIMACS version of bounded model checking problem into a
      file whose name depends on the system variable
      \envvar{bmc\_dimacs\_filename}. This feature is not allowed in
      combination of the option \commandopt{i}.}

\opt{-1}{Generates a single bounded model checking problem with fixed
      bound and loopback values, it does not iterate incrementing the
      value of the problem bound.}

\opt{-k \parameter{\natnum{\it bmc\_length}}}{\natnum{\it bmc\_length}
      is the maximum problem bound to be checked. Only natural numbers
      are valid values for this option. If no value is given the
      environment variable \envvar{\it bmc\_length} is considered
      instead.}

\opt{-l \parameter{\set{\it loopback}{\range{0}{max\_length-1},
      \range{-1}{bmc\_length}, X, *}}}{The {\it loopback} value may
      be: }

      \tabItem{a natural number in (0, {\it max\_length-1}). A
      positive sign (`+') can be also used as prefix of the
      number. Any invalid combination of length and loopback will be
      skipped during the generation/solving process.}

      \tabItem{a negative number in (-1, -{\it bmc\_length}). In this
      case {\it loopback} is considered a value relative to {\it
      max\_length}.  Any invalid combination of length and loopback
      will be skipped during the generation/solving process.}

      \tabItem{the symbol `\varvalue{X}', which means ``no loopback".}
      \tabItem{the symbol `\varvalue{*}', which means ``all possible
      loopbacks from zero to {\it length-1}". If no value is given the
      environment variable \envvar{\it bmc\_loopback} is considered
      instead.}

\end{cmdOpt}

\end{nusmvCommand}
