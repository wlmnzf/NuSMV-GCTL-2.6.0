% -*-latex-*-
%% BMC PICK STATE
\begin{nusmvCommand}{bmc\_pick\_state} {Picks a state from the set of initial states}

\cmdLine{bmc\_pick\_state [-h] [-v] [-c "constraint" | -s trace.state] [-r | -i [-a]]}

Chooses an element from the set of initial states, and makes it the
current state (replacing the old one). The chosen state is stored as
the first state of a new trace ready to be lengthened by steps states
by the \command{bmc\_simulate} command or the \command{bmc\_inc\_simulate} command.


\begin{cmdOpt}
\opt{-v}{Verbosely prints the generated trace}
\opt{-c \parameter{\it{constraint}}}{Set a constraint to narrow initial states.}
\opt{-s \parameter{\it{state}}}{Picks state from trace.state label.}
\opt{-r}{Randomly picks a state from the set of initial states.}
\opt{-i}{Enters simulation's interactive mode.}
\opt{-a}{Displays all the state variables (changed and unchanged)
  in the interactive session}
\end{cmdOpt}

\end{nusmvCommand}

%% BMC SIMULATE
\begin{nusmvCommand}{bmc\_simulate} {Generates a trace of the model from 0 (zero) to k}

\cmdLine{bmc\_simulate [-h] [-p | -v] [-r] [[-c "constraints"] | [-t
      "constraints"] ] [-k steps]}

\command{bmc\_simulate} does not require a specification to build the
problem, because only the model is used to build it.  The problem
length is represented by the \commandopt{k} command parameter, or by
its default value stored in the environment variable
\envvar{bmc\_length}.

\begin{cmdOpt}
\opt{-p}{Prints the generated trace (only changed variables).}
\opt{-v}{Prints the generated trace (all variables).}
\opt{-r}{Picks a state from a set of possible future states in a random way.}
\opt{-c \parameter{\it{constraint}}}{Performs a simulation in which
  computation is restricted to states satisfying those
  \code{constraints}. The desired sequence of states could not exist
  if such constraints were too strong or it may happen that at some
  point of the simulation a future state satisfying those constraints
  doesn't exist: in that case a trace with a number of states less
  than \code{steps} trace is obtained.  Note: \code{constraints} must
  be enclosed between double quotes \code{" "}.  The expression cannot
  contain \code{next} operators, and is automatically shifted by one
  state in order to constraint only the next steps}

\opt{-t \parameter{"constraints"}}{ Performs a simulation in which
  computation is restricted to states satisfying those
  \code{constraints}. The desired sequence of states could not exist
  if such constraints were too strong or it may happen that at some
  point of the simulation a future state satisfying those constraints
  doesn't exist: in that case a trace with a number of states less
  than \code{steps} trace is obtained.  Note: \code{constraints} must
  be enclosed between double quotes \code{" "}.  The expression can
  contain \code{next} operators, and is NOT automatically shifted by
  one state as done with option \code{-c}}

\opt{\natnum{-k steps}}{ Maximum length of the path according to the
  constraints.  The length of a trace could contain less than
  \code{steps} states: this is the case in which simulation stops in
  an intermediate step because it may not exist any future state
  satisfying those constraints. The default value is determined by the
  \varName{default\_simulation\_steps} environment variable}
\end{cmdOpt}

\end{nusmvCommand}

%% BMC INC SIMULATE
\begin{nusmvCommand}{bmc\_inc\_simulate} {Generates a trace of the model from 0 (zero) to k}

\cmdLine{bmc\_inc\_simulate [-h] [-p | -v] [-r | -i [-a]] [[-c "constraints"] |
    [-t "constraints"] ] [-k steps]}

Performs incremental simulation of the model.
\command{bmc\_inc\_simulate} does not require a specification to build the
problem, because only the model is used to build it.  The problem
length is represented by the \commandopt{k} command parameter, or by
its default value stored in the environment variable
\envvar{bmc\_length}.

\begin{cmdOpt}
\opt{-p}{Prints the generated trace (only changed variables).}
\opt{-v}{Prints the generated trace (all variables).}
\opt{-r}{Picks a state from a set of possible future states in a random way.}
\opt{-i}{Enters simulation's interactive mode.}
\opt{-a}{Displays all the state variables (changed and unchanged)
  in the interactive session}
\opt{-c \parameter{\it{constraint}}}{Performs a simulation in which
  computation is restricted to states satisfying those
  \code{constraints}. The desired sequence of states could not exist
  if such constraints were too strong or it may happen that at some
  point of the simulation a future state satisfying those constraints
  doesn't exist: in that case a trace with a number of states less
  than \code{steps} trace is obtained.  Note: \code{constraints} must
  be enclosed between double quotes \code{" "}.  The expression cannot
  contain \code{next} operators, and is automatically shifted by one
  state in order to constraint only the next steps}

\opt{-t \parameter{"constraints"}}{ Performs a simulation in which
  computation is restricted to states satisfying those
  \code{constraints}. The desired sequence of states could not exist
  if such constraints were too strong or it may happen that at some
  point of the simulation a future state satisfying those constraints
  doesn't exist: in that case a trace with a number of states less
  than \code{steps} trace is obtained.  Note: \code{constraints} must
  be enclosed between double quotes \code{" "}.  The expression can
  contain \code{next} operators, and is NOT automatically shifted by
  one state as done with option \code{-c}}

\opt{\natnum{-k steps}}{ Maximum length of the path according to the
  constraints.  The length of a trace could contain less than
  \code{steps} states: this is the case in which simulation stops in
  an intermediate step because it may not exist any future state
  satisfying those constraints. The default value is determined by the
  \varName{default\_simulation\_steps} environment variable}
\end{cmdOpt}

\end{nusmvCommand}


%% BMC SIMULATE CHECK FEASIBLE CONSTRAINTS
\begin{nusmvCommand}{bmc\_simulate\_check\_feasible\_constraints} {Checks feasability for the given constraints}

\cmdLine{bmc\_simulate\_check\_feasible\_constraints [-h] [-q] [-c "constr"]}

Checks if the given constraints are feasible for BMC simulation.

\begin{cmdOpt}
\opt{-q}{Prints the output in compact form.}
\opt{-c \parameter{\it{constr}}}{Specify one constraint whose
  feasability has to be checked (can be used multiple times, order is
  important to read the result)}
\end{cmdOpt}

\end{nusmvCommand}

