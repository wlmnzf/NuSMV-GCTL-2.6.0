% -*-latex-*-
\begin{nusmvCommand} {execute\_traces} {Executes complete traces on the model FSM}

\cmdLine{execute\_traces [-h] [-v] [-m | -o output-file]
  {-e engine [-a | trace\_number]}}

Executes traces stored in the Trace Manager.  If no trace is
specified, last registered trace is executed. Traces must be complete
in order to perform execution.

\begin{cmdOpt}
\opt{-v} { Verbosely prints traces execution steps.}

\opt{-a}{ Prints all the currently stored traces.}

\opt{-m}{ Pipes the output through the program specified by the
\shellvar{PAGER} shell variable if defined, else through the \unix
command \shellcommand{more}.}

\opt{-o \parameter{\filename{output-file}}}{Writes the output
  generated by the command to \filename{output-file}.}

\opt{-e \parameter{\filename{engine}}}{Selects an engine for trace
  re-execution. It must be one of 'bdd', 'sat' or 'smt'.}

\opt{\natnum{trace\_number}}{ The (ordinal) identifier number of the trace to
 be printed. This must be the last argument of the command. Omitting
 the trace number causes the most recently generated trace to be executed.}
\end{cmdOpt}

\end{nusmvCommand}
